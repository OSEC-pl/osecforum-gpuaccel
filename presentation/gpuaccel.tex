\documentclass[dvipsnames,table]{beamer}
\usepackage{polski}

\usetheme{Rochester}
\usecolortheme{orchid}

\usepackage{listings}
\usepackage{ucs}
\usepackage[utf8x]{inputenc}
\usepackage{wasysym}
\usepackage[normalem]{ulem}
\usepackage{amsmath}
\usepackage{hyperref}
\usepackage{tikzsymbols}

\setbeamertemplate{navigation symbols}{}
\setbeamertemplate{caption}[numbered]
\setbeamerfont{caption}{size=\scriptsize}
\setbeamercolor{framenote}{bg=OSEC-red!25}
\setbeamercolor{rednote}{bg=Red!25}
\setbeamercolor{palette primary}{use=structure,fg=white,bg=OSEC-red}
\setbeamercolor{palette secondary}{use=structure,fg=white,bg=OSEC-red2}

\setbeamertemplate{itemize item}{\scriptsize\raise1pt\hbox{\donotcoloroutermaths$\blacktriangleright$}}
\setbeamertemplate{itemize subitem}{\tiny\raise1pt\hbox{\donotcoloroutermaths$\bullet$}}
\setbeamertemplate{itemize subsubitem}{\tiny\raise1pt\hbox{\donotcoloroutermaths{--}}}

\setbeamertemplate{enumerate item}{\insertenumlabel.}
\setbeamertemplate{enumerate subitem}{\insertenumlabel.\insertsubenumlabel}
\setbeamertemplate{enumerate subsubitem}{\insertenumlabel.\insertsubenumlabel.\insertsubsubenumlabel}
\setbeamertemplate{enumerate mini template}{\insertenumlabel}

\setbeamercolor{itemize item}{fg=OSEC-red, bg=OSEC-red}
\setbeamercolor{itemize subitem}{fg=OSEC-red, bg=OSEC-red}
\setbeamercolor{itemize subsubitem}{fg=OSEC-red, bg=OSEC-red}

\setbeamercolor{section number projected}{fg=white,bg=OSEC-red}
\setbeamercolor{subsection number projected}{fg=white,bg=OSEC-red}
\setbeamercolor{button}{bg=OSEC-red,fg=white}

\setbeamertemplate{section in toc}[circle]
\setbeamertemplate{subsection in toc}[square]

\definecolor{OSEC-red}{RGB}{160,29,44}
\definecolor{OSEC-red2}{RGB}{177,76,12}
\hypersetup{colorlinks=true,linkcolor=white,urlcolor=OSEC-red}

\setlength{\tabcolsep}{8pt}
\renewcommand{\arraystretch}{1.2}

\newcommand{\tri}{$\triangleright$ }

\lstset{
   language=C,
   basicstyle=\tiny,
   breaklines=true,
   escapechar=\@,
   commentstyle=\color{NetBSD-orange}
}

\title{Gdy CPU nie wystarcza –- strategie optymalizacji oprogramowania z użyciem technologii GPU oraz FPGA}
\author{Radosław Kujawa -- radoslaw.kujawa@osec.pl}
\institute{OSEC}

\begin{document}

\begin{frame}
	\titlepage
\end{frame}

\begin{frame}
	\frametitle{Foo + bar}
\begin{itemize}
	\item Wydajność procesorów nie rośnie już tak szybko\dots
	\item Instancje maszyn wirtualnych {\em nie mogą} posiadać interfejsów przyłączonych bezpośrednio do sieci fizycznej (brak odp. bridge'y na węzłach obliczeniowych).
	\item Type driver {\tt vlan}/{\tt vxlan} -- dostęp do odseparowanego segmentu sieci fizycznej.
	\item Warstwa 2 - sterowniki ML2 (,,mechanism driver'' np. Open vSwitch).
\end{itemize}
\end{frame}

\begin{frame}
	\frametitle{Nvidia CUDA}
\begin{itemize}
	\item - Closed source.
	\item - Vendor lock-in.
	\item - Dobre materiały edukacyjne.
	\item Duża popularność.
	\item Gotowe biblioteki blas, lapack itd.
\end{itemize}
\end{frame}

\begin{frame}
	\frametitle{OpenCL}
\begin{itemize}
	\item Jak OpenGL, tylko do obliczeń.
	\item Analogiczna architektura oparta o ICD, dostarczany przez dostawcę sprzętu.
\end{itemize}
\end{frame}

\begin{frame}
\begin{itemize}
	\item Badania naukowe wymagające obliczeń równoległych.
	\item Wydobycie walut kryptograficznych.
	\item Akceleracja przeglądarek z użyciem GPU:
	\begin{itemize}
		\item Chrome
		\item Firefox
	\end{itemize}
\end{itemize}
\end{frame}

\begin{frame}
	\frametitle{Popularne implementacje OpenCL}
\begin{itemize}
	\item AMDGPU Pro.
	\item Mesa Clover (AMD).
	\item Nvidia (bazuje na CUDA).
	\item Beignet (Intel).
	\item pocl (CPU).
	\item Intel OpenCL (GPU, Xeon Phi).
\end{itemize}
\end{frame}

\begin{frame}
	\frametitle{OpenACC}
\begin{itemize}
	\item Deklaracje w zwykłym kodzie C/C++.
	\item nvptx (CUDA)
	\item gcc dla nvptx (open source)
\end{itemize}
\end{frame}

\begin{frame}
	\frametitle{GPGPU i FPGA w chmurze}
\begin{itemize}
	\item Amazon F1.
	\item GPU w Amazon.
\end{itemize}
\end{frame}

\begin{frame}
	\frametitle{Algorytmy...}
\begin{itemize}
	\item Od początku istnienia komputerów, większość algorytmów była projektowana jako algrytmy sekwencyjne.
	\item Wyodrębnienie części algorytmu, która może wykonywać się równolegle.
\end{itemize}
\end{frame}

\begin{frame}
	\frametitle{Konwersja algorytmu sekwencyjnego w równoległy}
\begin{itemize}
	\item foo
\end{itemize}
\end{frame}

% Ian Foster
% Designing and Building Parallel Programs
% http://www.mcs.anl.gov/~itf/dbpp/text/book.html

% Uzi Vishkin
% Thinking in Parallel: Some Basic Data-Parallel Algorithms and Techniques
% http://legacydirs.umiacs.umd.edu/~vishkin/PUBLICATIONS/classnotes.pdf


%\begin{frame}
%\frametitle{Architektura z centralnym punktem dostępu do sieci zewnętrznej}
%\includegraphics[width=1.00\textwidth]{img-neutron-tenantonly.pdf}
%\end{frame}

\begin{frame}
	\frametitle{Biblioteki wyższego poziomu}
\begin{itemize}
	\item Biblioteki wyższego poziomu.
	\item Łatwiejsze w obsłudze, niższy próg wejścia niż do CUDA/OpenCL.
	\item Wiele gotowych bibliotek dla specyficznych dziedzin programowania.
	\begin{itemize}
		\item OpenCV.
		\item Basic Linear Algebra Subprograms (BLAS).
		\item \ldots 
	\end{itemize}
\end{itemize}
\end{frame}

\begin{frame}
	\frametitle{Przykład: BLAS}
\begin{itemize}
	\item \href{http://www.netlib.org/blas/}{CBLAS (CPU)}
	\item \href{https://docs.nvidia.com/cuda/cublas/index.html}{cuBLAS (CUDA)} 
	\item \href{https://github.com/CNugteren/CLBlast}{CLBlast (OpenCL)}
	\item \href{https://developer.apple.com/documentation/accelerate/blas}{Accelerate.framework (CPU SIMD, macOS)}
	\item \ldots
\end{itemize}
\end{frame}

\begin{frame}[fragile]
\frametitle{Przykład: Mnożenie macierzy za pomocą  BLAS}
\begin{itemize}
	\item Funkcja {\tt sgemm} (poj. prezycja) / {\tt dgemm} (podwójna precyzja).
\end{itemize}
\begin{lstlisting}
void cblas_sgemm(const CBLAS_LAYOUT layout, const CBLAS_TRANSPOSE TransA,
	const CBLAS_TRANSPOSE TransB, const int m, const int n,
	const int k, const float alpha, const float  *A,
	const int lda, const float  *B, const int ldb,
	const float beta, float  *C, const int ldc);
\end{lstlisting}
\[ C \leftarrow \alpha op(A) op(B) + \beta C \]
\begin{itemize}
	\item $op(X)$ to $op(X) = X$, lub $op(X) = X^T$ lub $op(X) = X^H$.
	\item $\alpha$ i $\beta$ są skalarami.
	\item $A$, $B$ i $C$ są macierzami.
	\item $op(A)$ jest macierzą $m$ x $k$.
	\item $op(B)$ jest macierzą $k$ x $n$.
	\item $C$ jest macierzą $m$ x $n$.
\end{itemize}
\end{frame}

\begin{frame}[fragile]
	\frametitle{foo}

\[
 \begin{bmatrix}
  a & b & c \\
  d & e & f \\
  g & h & i
 \end{bmatrix}
\]
	\begin{itemize}
	\item foo
	\end{itemize}
\end{frame}

\begin{frame}
\frametitle{Koniec\ldots}
\begin{center}
\includegraphics[scale=0.5]{img-oseclogo.png}

Dziękuje!

Czy są pytania?

\end{center}
\end{frame}
\end{document}

